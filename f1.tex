\relax
\input /home/bdg/Templates/center.tex
\input /home/bdg/Templates/nsets.tex

\bf
\startcenter
\'Algebra Universal e Categorias

Bruno Dias da Gi\~ao, A96544

Licenciatura em Ci\^encias da Computa\c{c}\~ao
\stopcenter

\

\startcenter
\sl Manuscrito realizado usando \TeX.
\stopcenter

\

\bf
Resolu\c{c}\~ao de Exerc\'{\i}cios

\

1. Reticulados

1.1.
\rm

P$_1$ \'{e} um reticulado.

\bf Demonstra\c{c}\~ao \rm

Lembrar-nos-emos da defini\c{c}\~{a}o de reticulado,

\startcenter
\it R diz-se reticulado se para qq $x,y \in R$, temos que
            inf$\{x,y\}$ e sup$\{x,y\}$ existem. \rm
\stopcenter

Ora, sendo assim, verificaremos para os pontos a, b e c.

\

\startcenter
inf$\{a,b\} =$ inf$\{a,c\} =$ inf$\{b,c\} = 0$

sup$\{a,b\} =$ sup$\{a,c\} =$ sup$\{b,c\} = 1$
\stopcenter

\

Logo, temos que, efetivamente, P$_1$ \'e um reticulado.\ \ \ \ \ \ QED

\

P$_2$ n\~{a}o \'{e} um reticulado.

\bf Demonstra\c{c}\~{a}o \rm

Comecemos por estudar os pontos que n\~ao est\~ao em rela\c{c}\~{a}o entre si, ou
seja, temos os seguintes pares:

\startcenter
$(a,b)$ e $(c,d)$
\stopcenter

Ora, temos que Maj$\{a,b\} = \{c,d,1\}$ e Min$\{c,d\} = \{a,b,0\}$.

Sendo assim, temos que:
\startcenter
sup$\{a,b\} \buildrel \rm def \over =$ min$\{$Maj$\{a,b\}\} \notin$ P$_2$
\stopcenter
e, analogamente,
\startcenter
inf$\{c,d\} \buildrel \rm def \over =$ max$\{$Min$\{c,d\}\} \notin$ P$_2$.
\stopcenter

Logo, P$_2$ n\~{a}o \'{e} um reticulado.\ \ \ \ \ \ \ \ QED

\bf
1.2.a
\rm $(\bbb N, \mid\,)$ \'e reticulado?

Pretendemos demonstrar que o c.p.o. formado pelos n\'umeros naturais, $\bbb N$, sobre a
rela\c{c}\~{a}o divide, $\mid\,$, \'{e} um reticulado.

\bf Demonstra\c{c}\~ao \rm



Logo, $(\bbb N, \mid\,)$ \'{e} um reticulado.

\vfil \break

\bf
1.3
\rm Toda a cadeia \'e um reticulado?

\bf Demonstra\c{c}\~ao

\bye
